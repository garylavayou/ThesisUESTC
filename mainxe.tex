\documentclass[doctor]{thesis-uestc}
\setCJKmonofont{Courier New}
\hypersetup{
  colorlinks=true,
  linkcolor=blue,
  urlcolor=cyan
}

\title{关于模板的用法示例和问题测试}
\author{王刚}
\studentid{201311260112}

\begin{document}

\thesisglossarylist

\thesischapterexordium
本文档用于说明模板中一些命令,以及展示论文写作中常用功能。
用于论文写作的样例文档,请参考``main.tex''和``main\_multifile.tex''。

\chapter{数学公式}

数学字体采用\texttt{newtxmath}代替原模板采用的\texttt{mathptmx},使手写体更加工整。
如果不喜欢当前类型,可以选择原模板的字体设置(参考\href{https://github.com/x-magus/ThesisUESTC/issues/58}{x-magus/ThesisUESTC {\#}58})。
另外,采用\texttt{newtxmath}也可以避免\texttt{mathptmx}不支持加粗字体(不会出错,但会产生警告)。

\section{字母和符号}

小写字母:\(abcdefghijklmnopqrstuvwxyz\)

大写字母:\(ABCDEFGHIJKLMNOPQRSTUVWXYZ\)

手写体:~~~~\(\mathcal{ABCDEFGHIJKLMNOPQRSTUVWXYZ}\)

双线:~~~~~~~~\(\mathbb{ABCDEFGHIJKLMNOPQRSTUVWXYZ}\)

粗体正体:\(\mathbf{ABCDEFGHIJKLMNOPQRSTUVWXYZ}\)

粗体斜体:\(\bm{ABCDEFGHIJKLMNOPQRSTUVWXYZ}\)

\chapter{术语缩略词}

%% 定义术语
\newglossaryentry{Linux}
{
  name=Linux,
  description={is a generic term referring to the family of Unix-like
    computer operating systems that use the Linux kernel},
  plural=Linuces
}
%% 定义术语。
\newacronym[description=逻辑卷管理器]{lvm}{LVM}{Logical Volume Manager}

%% 使用术语和缩略词
\gls{Linux}是现代操作系统之一。
\gls{lvm}是计算机磁盘管理的重要功能,(\gls{lvm})使得多个磁盘可同时工作在同一计算机上。

\end{document}